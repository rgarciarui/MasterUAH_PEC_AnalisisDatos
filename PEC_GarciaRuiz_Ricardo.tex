\documentclass[]{article}
\usepackage{lmodern}
\usepackage{amssymb,amsmath}
\usepackage{ifxetex,ifluatex}
\usepackage{fixltx2e} % provides \textsubscript
\ifnum 0\ifxetex 1\fi\ifluatex 1\fi=0 % if pdftex
  \usepackage[T1]{fontenc}
  \usepackage[utf8]{inputenc}
\else % if luatex or xelatex
  \ifxetex
    \usepackage{mathspec}
  \else
    \usepackage{fontspec}
  \fi
  \defaultfontfeatures{Ligatures=TeX,Scale=MatchLowercase}
\fi
% use upquote if available, for straight quotes in verbatim environments
\IfFileExists{upquote.sty}{\usepackage{upquote}}{}
% use microtype if available
\IfFileExists{microtype.sty}{%
\usepackage{microtype}
\UseMicrotypeSet[protrusion]{basicmath} % disable protrusion for tt fonts
}{}
\usepackage[margin=1in]{geometry}
\usepackage{hyperref}
\hypersetup{unicode=true,
            pdftitle={PEC\_Analisis Datos R},
            pdfauthor={Ricardo Garcia Ruiz},
            pdfborder={0 0 0},
            breaklinks=true}
\urlstyle{same}  % don't use monospace font for urls
\usepackage{graphicx,grffile}
\makeatletter
\def\maxwidth{\ifdim\Gin@nat@width>\linewidth\linewidth\else\Gin@nat@width\fi}
\def\maxheight{\ifdim\Gin@nat@height>\textheight\textheight\else\Gin@nat@height\fi}
\makeatother
% Scale images if necessary, so that they will not overflow the page
% margins by default, and it is still possible to overwrite the defaults
% using explicit options in \includegraphics[width, height, ...]{}
\setkeys{Gin}{width=\maxwidth,height=\maxheight,keepaspectratio}
\IfFileExists{parskip.sty}{%
\usepackage{parskip}
}{% else
\setlength{\parindent}{0pt}
\setlength{\parskip}{6pt plus 2pt minus 1pt}
}
\setlength{\emergencystretch}{3em}  % prevent overfull lines
\providecommand{\tightlist}{%
  \setlength{\itemsep}{0pt}\setlength{\parskip}{0pt}}
\setcounter{secnumdepth}{5}
% Redefines (sub)paragraphs to behave more like sections
\ifx\paragraph\undefined\else
\let\oldparagraph\paragraph
\renewcommand{\paragraph}[1]{\oldparagraph{#1}\mbox{}}
\fi
\ifx\subparagraph\undefined\else
\let\oldsubparagraph\subparagraph
\renewcommand{\subparagraph}[1]{\oldsubparagraph{#1}\mbox{}}
\fi

%%% Use protect on footnotes to avoid problems with footnotes in titles
\let\rmarkdownfootnote\footnote%
\def\footnote{\protect\rmarkdownfootnote}

%%% Change title format to be more compact
\usepackage{titling}

% Create subtitle command for use in maketitle
\newcommand{\subtitle}[1]{
  \posttitle{
    \begin{center}\large#1\end{center}
    }
}

\setlength{\droptitle}{-2em}

  \title{PEC\_Analisis Datos R}
    \pretitle{\vspace{\droptitle}\centering\huge}
  \posttitle{\par}
    \author{Ricardo Garcia Ruiz}
    \preauthor{\centering\large\emph}
  \postauthor{\par}
      \predate{\centering\large\emph}
  \postdate{\par}
    \date{17 de enero, 2019}


\begin{document}
\maketitle

{
\setcounter{tocdepth}{4}
\tableofcontents
}
\section{Ejercicio 1 (40\%)}\label{ejercicio-1-40}

\begin{quote}
El conjunto de datos bikes2016.csv contiene información sobre el número
de personas que circulan en bicicleta por cada uno de los distritos de
la ciudad de Montreal a lo largo del año 2016
\url{http://donnees.ville.montreal.qc.ca/dataset/velos-comptage}. En
este conjunto de datos, las filas representan los dias del año y las
columnas cada uno de los distritos. La columna 2 contiene un timestamp
que vamos a ignorar. Sobre este conjunto de datos:
\end{quote}

\subsection{(5 puntos) Leer el fichero bikes2016.csv como un
dataframe.}\label{puntos-leer-el-fichero-bikes2016.csv-como-un-dataframe.}

\subsection{\texorpdfstring{(5 puntos) Eliminar la columna
\textbf{Timestamp} del
dataframe.}{(5 puntos) Eliminar la columna Timestamp del dataframe.}}\label{puntos-eliminar-la-columna-timestamp-del-dataframe.}

\subsection{(10 puntos) Calcular el porcentaje de missing values en cada
una de las
columnas.}\label{puntos-calcular-el-porcentaje-de-missing-values-en-cada-una-de-las-columnas.}

\subsection{(10 puntos) Identificar cuales son las variables que están
contenidas en el data.frame. A continuación, transformar ese data.frame
para que cada columna represente cada una de las variables. Usar los
nombres Date, District y
N.}\label{puntos-identificar-cuales-son-las-variables-que-estan-contenidas-en-el-data.frame.-a-continuacion-transformar-ese-data.frame-para-que-cada-columna-represente-cada-una-de-las-variables.-usar-los-nombres-date-district-y-n.}

\subsection{(10 puntos) Calcular el total de personas que pasa por cada
uno de los distritos a lo largo de todo el
año.}\label{puntos-calcular-el-total-de-personas-que-pasa-por-cada-uno-de-los-distritos-a-lo-largo-de-todo-el-ano.}

\subsection{\texorpdfstring{(10 puntos) Completar los missing values del
atributo N con la media del resto de datos de esa variable pero agrupado
de acuerdo a la variable
\textbf{District}.}{(10 puntos) Completar los missing values del atributo N con la media del resto de datos de esa variable pero agrupado de acuerdo a la variable District.}}\label{puntos-completar-los-missing-values-del-atributo-n-con-la-media-del-resto-de-datos-de-esa-variable-pero-agrupado-de-acuerdo-a-la-variable-district.}

\subsection{\texorpdfstring{(10 puntos) Crear tres nuevas variables en
el data.frame \textbf{(Day, Month, Year )} que contengan la información
del día, mes y año
respectivamente.}{(10 puntos) Crear tres nuevas variables en el data.frame (Day, Month, Year ) que contengan la información del día, mes y año respectivamente.}}\label{puntos-crear-tres-nuevas-variables-en-el-data.frame-day-month-year-que-contengan-la-informacion-del-dia-mes-y-ano-respectivamente.}

\subsection{(20 puntos) Realizar un gráfico de barras del número de
ciclistas para cada día de la semana en cada uno de los seis districtos
con más ciclistas (usando
facetas)}\label{puntos-realizar-un-grafico-de-barras-del-numero-de-ciclistas-para-cada-dia-de-la-semana-en-cada-uno-de-los-seis-districtos-con-mas-ciclistas-usando-facetas}

\subsection{\texorpdfstring{(20 puntos) Realizar un único gráfico con la
evolución diaria del número de ciclistas en el mes de Enero para los
districtos \textbf{Berry1}, \textbf{University}, \textbf{Boyer} y
\textbf{ParcB}.}{(20 puntos) Realizar un único gráfico con la evolución diaria del número de ciclistas en el mes de Enero para los districtos Berry1, University, Boyer y ParcB.}}\label{puntos-realizar-un-unico-grafico-con-la-evolucion-diaria-del-numero-de-ciclistas-en-el-mes-de-enero-para-los-districtos-berry1-university-boyer-y-parcb.}

\section{Ejercicio 2 (60\%)}\label{ejercicio-2-60}

\begin{quote}
El conjunto de datos titanic.csv contiene información sobre los
pasajeros del barco. Este conjunto de datos se ha utilizado para tratar
de predecir la supervivencia de un pasajero en base a otra serie de
variables como edad, sexo, o la clase del billete. Ver por ejemplo:
\url{https://www.kaggle.com/c/titanic}. Cada una de las variables del
fichero contiene la siguiente información:
\end{quote}

\begin{itemize}
\tightlist
\item
  \textbf{survival:} Supervivencia (0 = No; 1 = Yes)
\item
  \textbf{pclass:} Clase de pasajero (1, 2, 3)
\item
  \textbf{name:} Nombre
\item
  \textbf{sex:} Sexo
\item
  \textbf{age:} Edad
\item
  \textbf{sibsp:} Número de hermanos/esposos/as a bordo.
\item
  \textbf{parch:} Número de padres/hijos a bordo
\item
  \textbf{ticket:} Número de ticket
\item
  \textbf{fare:} Coste del billete
\item
  \textbf{cabin:} Cabina
\item
  \textbf{embarked:} Puerto de embarque
\end{itemize}

Con el fichero de datos anterior:

\subsection{\texorpdfstring{(4 puntos) Leer el fichero
\textbf{titanic.csv} como un
dataframe.}{(4 puntos) Leer el fichero titanic.csv como un dataframe.}}\label{puntos-leer-el-fichero-titanic.csv-como-un-dataframe.}

\subsection{(2 puntos) Calcular el porcentaje de pasajeros que
sobrevivió.}\label{puntos-calcular-el-porcentaje-de-pasajeros-que-sobrevivio.}

\subsection{(4 puntos) Calcular el porcentaje de missing values en cada
uno de los
atributos.}\label{puntos-calcular-el-porcentaje-de-missing-values-en-cada-uno-de-los-atributos.}

\subsection{\texorpdfstring{(2 puntos) Eliminar la variable
\textbf{Cabin} del
dataframe.}{(2 puntos) Eliminar la variable Cabin del dataframe.}}\label{puntos-eliminar-la-variable-cabin-del-dataframe.}

\subsection{\texorpdfstring{(8 puntos) Crear una nueva variable
\textbf{Title} a partir de \textbf{Name} con los valores \textbf{Master}
(hombre soltero), \textbf{Miss} (mujer soltera), \textbf{Mr.} (hombre
casado), \textbf{Mrs.} (mujer casada) y Otro a partir de la variable
\textbf{nombre}. Es importante tener en cuenta que el título
\textbf{Miss} está en ocasiones codificado con su abreviatura en frances
\textbf{Mlle} (mademoiselle) y lo mismo ocurre con \textbf{Mrs.}, que en
ocasiones aparece como \textbf{Ms.} ó \textbf{Mme}
(madame).}{(8 puntos) Crear una nueva variable Title a partir de Name con los valores Master (hombre soltero), Miss (mujer soltera), Mr. (hombre casado), Mrs. (mujer casada) y Otro a partir de la variable nombre. Es importante tener en cuenta que el título Miss está en ocasiones codificado con su abreviatura en frances Mlle (mademoiselle) y lo mismo ocurre con Mrs., que en ocasiones aparece como Ms. ó Mme (madame).}}\label{puntos-crear-una-nueva-variable-title-a-partir-de-name-con-los-valores-master-hombre-soltero-miss-mujer-soltera-mr.-hombre-casado-mrs.-mujer-casada-y-otro-a-partir-de-la-variable-nombre.-es-importante-tener-en-cuenta-que-el-titulo-miss-esta-en-ocasiones-codificado-con-su-abreviatura-en-frances-mlle-mademoiselle-y-lo-mismo-ocurre-con-mrs.-que-en-ocasiones-aparece-como-ms.-o-mme-madame.}

\subsection{\texorpdfstring{(4 puntos) Explorar la relación entre las
variables \textbf{Age} y la nueva variable \textbf{Title} mediante un
boxplot para cada uno de los valores de la misma. ¿Tienen alguna
relación?.}{(4 puntos) Explorar la relación entre las variables Age y la nueva variable Title mediante un boxplot para cada uno de los valores de la misma. ¿Tienen alguna relación?.}}\label{puntos-explorar-la-relacion-entre-las-variables-age-y-la-nueva-variable-title-mediante-un-boxplot-para-cada-uno-de-los-valores-de-la-misma.-tienen-alguna-relacion.}

\subsection{\texorpdfstring{(4 puntos) Explorar la relación entre
\textbf{Age}, \textbf{Pclass} y \textbf{Title} en varios gráficos de
dispersión con colores, donde el color representa la supervivencia
(Pista: usar
facetas).}{(4 puntos) Explorar la relación entre Age, Pclass y Title en varios gráficos de dispersión con colores, donde el color representa la supervivencia (Pista: usar facetas).}}\label{puntos-explorar-la-relacion-entre-age-pclass-y-title-en-varios-graficos-de-dispersion-con-colores-donde-el-color-representa-la-supervivencia-pista-usar-facetas.}

\subsection{\texorpdfstring{(8 puntos) Completar los missing values del
atributo \textbf{Age} con la mediana del resto de datos de esa variable
pero agrupado de acuerdo a las variables \textbf{Pclass} y
\textbf{Title}.}{(8 puntos) Completar los missing values del atributo Age con la mediana del resto de datos de esa variable pero agrupado de acuerdo a las variables Pclass y Title.}}\label{puntos-completar-los-missing-values-del-atributo-age-con-la-mediana-del-resto-de-datos-de-esa-variable-pero-agrupado-de-acuerdo-a-las-variables-pclass-y-title.}

\subsection{(2 puntos) Después de realizar las operaciones anteriores,
eliminar ahora cualquier fila que tenga al menos un
NA.}\label{puntos-despues-de-realizar-las-operaciones-anteriores-eliminar-ahora-cualquier-fila-que-tenga-al-menos-un-na.}

\subsection{(2 puntos) Calcular la probabilidad de supervivencia en base
al género (Sex). ¿Qué conclusión(es) obtienes del
resultado?}\label{puntos-calcular-la-probabilidad-de-supervivencia-en-base-al-genero-sex.-que-conclusiones-obtienes-del-resultado}

\subsection{\texorpdfstring{(2 puntos) Calcular la probabilidad de
supervivencia en base a la edad (\textbf{Age}). ¿Te parecen fácilmente
interpretables estos
resultados?}{(2 puntos) Calcular la probabilidad de supervivencia en base a la edad (Age). ¿Te parecen fácilmente interpretables estos resultados?}}\label{puntos-calcular-la-probabilidad-de-supervivencia-en-base-a-la-edad-age.-te-parecen-facilmente-interpretables-estos-resultados}

\subsection{\texorpdfstring{(4 puntos) Crea una nueva variable
\textbf{Decade} en el dataframe que contenga la década de la edad de los
pasajeros y repite el análisis del apartado anterior sobre esta nueva
variable. ¿Qué conclusión(es) obtienes del resultado? Pista: función
cut.}{(4 puntos) Crea una nueva variable Decade en el dataframe que contenga la década de la edad de los pasajeros y repite el análisis del apartado anterior sobre esta nueva variable. ¿Qué conclusión(es) obtienes del resultado? Pista: función cut.}}\label{puntos-crea-una-nueva-variable-decade-en-el-dataframe-que-contenga-la-decada-de-la-edad-de-los-pasajeros-y-repite-el-analisis-del-apartado-anterior-sobre-esta-nueva-variable.-que-conclusiones-obtienes-del-resultado-pista-funcion-cut.}

\subsection{\texorpdfstring{(4 puntos) Convertir la variable
\textbf{Survived} a un factor con los niveles Yes si ha sobrevivido y No
en caso
contrario.}{(4 puntos) Convertir la variable Survived a un factor con los niveles Yes si ha sobrevivido y No en caso contrario.}}\label{puntos-convertir-la-variable-survived-a-un-factor-con-los-niveles-yes-si-ha-sobrevivido-y-no-en-caso-contrario.}

\subsection{\texorpdfstring{(4 puntos) Ver la relación entre la
supervivencia y la nueva variable Title con un gráfico de barras. En el
caso del valor \textbf{Otros} de la variable \textbf{Title}, ¿nos
proporciona este alguna información sobre la supervivencia?. ¿A qué se
debe?}{(4 puntos) Ver la relación entre la supervivencia y la nueva variable Title con un gráfico de barras. En el caso del valor Otros de la variable Title, ¿nos proporciona este alguna información sobre la supervivencia?. ¿A qué se debe?}}\label{puntos-ver-la-relacion-entre-la-supervivencia-y-la-nueva-variable-title-con-un-grafico-de-barras.-en-el-caso-del-valor-otros-de-la-variable-title-nos-proporciona-este-alguna-informacion-sobre-la-supervivencia.-a-que-se-debe}

\subsection{(4 puntos) Crea dos nuevas variables en el dataframe con la
siguiente
información:}\label{puntos-crea-dos-nuevas-variables-en-el-dataframe-con-la-siguiente-informacion}

\begin{itemize}
\tightlist
\item
  \textbf{Familysize:} número total de parientes incluyendo al propio
  pasajero.
\item
  \textbf{Sigleton:} valor lógico indicando con valor TRUE si el
  pasajero viaja solo y FALSE en caso contrario.
\end{itemize}

\subsection{\texorpdfstring{(4 puntos) Realizar un gráfico de puntos de
la variable \textbf{Age} sobre \textbf{Fare}, coloreado por los valores
de la variable
\textbf{Survived}.}{(4 puntos) Realizar un gráfico de puntos de la variable Age sobre Fare, coloreado por los valores de la variable Survived.}}\label{puntos-realizar-un-grafico-de-puntos-de-la-variable-age-sobre-fare-coloreado-por-los-valores-de-la-variable-survived.}

\subsection{(2 puntos) Realizar un histograma para ver la distribución
de las
edades.}\label{puntos-realizar-un-histograma-para-ver-la-distribucion-de-las-edades.}

\subsection{\texorpdfstring{(4 puntos) Representar en un gráfico de
barras el número de pasajeros que han sobrevivido para cada uno de los
valores de las variables \textbf{Sex} y
\textbf{Pclass}.}{(4 puntos) Representar en un gráfico de barras el número de pasajeros que han sobrevivido para cada uno de los valores de las variables Sex y Pclass.}}\label{puntos-representar-en-un-grafico-de-barras-el-numero-de-pasajeros-que-han-sobrevivido-para-cada-uno-de-los-valores-de-las-variables-sex-y-pclass.}

\subsection{(4 puntos) Cuenta el número de pasajeros por tamaño de
familia y clase. Por ejemplo, cuántos pasajeros de primera clase
pertenecen a una familia de tamaño 4. El resultado debe ser un dataframe
con la información para todas las posibles combinaciones de clase del
billete y tamaño de
familia.}\label{puntos-cuenta-el-numero-de-pasajeros-por-tamano-de-familia-y-clase.-por-ejemplo-cuantos-pasajeros-de-primera-clase-pertenecen-a-una-familia-de-tamano-4.-el-resultado-debe-ser-un-dataframe-con-la-informacion-para-todas-las-posibles-combinaciones-de-clase-del-billete-y-tamano-de-familia.}

\subsection{\texorpdfstring{(4 puntos) Representar, en un mismo gráfico,
dos histogramas de la variable \textbf{Age}, uno para los pasajeros con
sexo masculino y otro para los pasajeros con sexo femenino. En caso de
que se solapen los histogramas, usar colores con
transparencias.}{(4 puntos) Representar, en un mismo gráfico, dos histogramas de la variable Age, uno para los pasajeros con sexo masculino y otro para los pasajeros con sexo femenino. En caso de que se solapen los histogramas, usar colores con transparencias.}}\label{puntos-representar-en-un-mismo-grafico-dos-histogramas-de-la-variable-age-uno-para-los-pasajeros-con-sexo-masculino-y-otro-para-los-pasajeros-con-sexo-femenino.-en-caso-de-que-se-solapen-los-histogramas-usar-colores-con-transparencias.}

\subsection{\texorpdfstring{(4 puntos) Leer el fichero
\textbf{titanic2.csv}, que contiene información adicional sobre los
pasajeros del
barco:}{(4 puntos) Leer el fichero titanic2.csv, que contiene información adicional sobre los pasajeros del barco:}}\label{puntos-leer-el-fichero-titanic2.csv-que-contiene-informacion-adicional-sobre-los-pasajeros-del-barco}

\begin{itemize}
\tightlist
\item
  \textbf{boat:} identificador del bote salvavidas
\item
  \textbf{body:} identificador del cuerpo
\item
  \textbf{home.dest:} Origen/destino
\end{itemize}

\subsection{(4 puntos) Para unificar estos dos dataframes, parecería
buena opción utilizar la variable name como clave. Determina si esta
variable es única por pasajero, mostrando el número de nombres
diferentes repetidos. En caso de existir varios pasajeros con el mismo
nombre, listar aquellas filas del dataframe inicial en las que el nombre
del pasajero esté
repetido}\label{puntos-para-unificar-estos-dos-dataframes-pareceria-buena-opcion-utilizar-la-variable-name-como-clave.-determina-si-esta-variable-es-unica-por-pasajero-mostrando-el-numero-de-nombres-diferentes-repetidos.-en-caso-de-existir-varios-pasajeros-con-el-mismo-nombre-listar-aquellas-filas-del-dataframe-inicial-en-las-que-el-nombre-del-pasajero-este-repetido}

\subsection{(6 puntos) Combina ambos dataframes utilizando la
combinación del nombre y el número de billete, manteniendo las mismas
filas que el dataframe
original.}\label{puntos-combina-ambos-dataframes-utilizando-la-combinacion-del-nombre-y-el-numero-de-billete-manteniendo-las-mismas-filas-que-el-dataframe-original.}

\subsection{(4 puntos) ¿Qué porcentaje de los pasajeros que sobrevivió
tiene asociado un identificador del bote
salvavidas?}\label{puntos-que-porcentaje-de-los-pasajeros-que-sobrevivio-tiene-asociado-un-identificador-del-bote-salvavidas}

\subsection{(6 puntos) Separar el conjunto anterior de datos en dos
subconjuntos disjuntos de forma aleatoria, el primero conteniendo un
70\% de los datos y el segundo un 30\%. Los resultados tienen que estar
contenidos en dos
dataframes}\label{puntos-separar-el-conjunto-anterior-de-datos-en-dos-subconjuntos-disjuntos-de-forma-aleatoria-el-primero-conteniendo-un-70-de-los-datos-y-el-segundo-un-30.-los-resultados-tienen-que-estar-contenidos-en-dos-dataframes}


\end{document}
